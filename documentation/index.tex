\documentclass{article}
\usepackage{geometry}
\usepackage{amsmath}
\usepackage{hyperref}

\title{Application for Transcribing Grand Piano Recordings to Sheet Music in PDF}
\author{Your Name}
\date{\today}

\begin{document}

\maketitle

\tableofcontents

\section{Topic Analysis}

\subsection{Overview}
This project involves creating an application designed to transcribe grand piano recordings into sheet music and export the results as a PDF. This involves the use of audio processing, music transcription, and PDF generation technologies.

\subsection{Objectives}
\begin{itemize}
    \item To accurately transcribe piano recordings into readable sheet music.
    \item To provide a user-friendly interface for uploading audio files and exporting sheet music.
    \item To offer customizable options for the transcription process to suit different user needs.
\end{itemize}

\subsection{Key Components}
\begin{itemize}
    \item \textbf{Audio Processing:} Capturing and processing the audio signals from the grand piano recordings.
    \item \textbf{Music Transcription:} Converting the processed audio signals into musical notation.
    \item \textbf{PDF Generation:} Formatting the musical notation into a PDF document that can be easily printed or shared.
    \item \textbf{User Interface:} Allowing users to interact with the application, upload audio files, customize transcription settings, and download the resulting sheet music.
\end{itemize}

\section{Functional Requirements}

\subsection{Core Functionalities}
\begin{itemize}
    \item \textbf{Audio File Upload:}
    \begin{itemize}
        \item Users can upload audio files in various formats (e.g., MP3, WAV).
    \end{itemize}
    \item \textbf{Audio Processing:}
    \begin{itemize}
        \item The system processes the uploaded audio to identify notes, rhythms, and dynamics.
    \end{itemize}
    \item \textbf{Transcription:}
    \begin{itemize}
        \item The application converts the processed audio data into musical notation.
        \item Users can view the transcription in real-time as it is being processed.
    \end{itemize}
    \item \textbf{Editing Tools:}
    \begin{itemize}
        \item Users can manually adjust the transcription for accuracy.
        \item Tools for adding/removing notes, adjusting rhythms, and adding dynamics and articulations.
    \end{itemize}
    \item \textbf{PDF Export:}
    \begin{itemize}
        \item Users can export the final sheet music as a PDF file.
        \item Options to include title, composer, and other metadata in the PDF.
    \end{itemize}
    \item \textbf{Playback Feature:}
    \begin{itemize}
        \item Users can listen to the transcription to verify accuracy.
        \item Synchronized playback with the sheet music display.
    \end{itemize}
\end{itemize}

\subsection{Advanced Functionalities}
\begin{itemize}
    \item \textbf{Customization:}
    \begin{itemize}
        \item Users can adjust settings for the transcription process, such as tempo detection, key signature, and time signature.
    \end{itemize}
    \item \textbf{Multi-Language Support:}
    \begin{itemize}
        \item The interface supports multiple languages for wider accessibility.
    \end{itemize}
    \item \textbf{Cloud Integration:}
    \begin{itemize}
        \item Option to save transcriptions to cloud storage services like Google Drive or Dropbox.
    \end{itemize}
    \item \textbf{Collaboration:}
    \begin{itemize}
        \item Users can share transcription projects with others for collaboration.
    \end{itemize}
\end{itemize}

\section{Non-Functional Requirements}

\subsection{Performance Requirements}
\begin{itemize}
    \item \textbf{Speed:} The transcription process should be completed within a reasonable time frame (e.g., under 5 minutes for a 5-minute recording).
    \item \textbf{Accuracy:} The transcription should accurately reflect the notes, rhythms, and dynamics of the original recording with a high degree of precision (e.g., over 90\% accuracy).
\end{itemize}

\subsection{Usability Requirements}
\begin{itemize}
    \item \textbf{User Interface:} The interface should be intuitive and easy to navigate, even for users with limited technical skills.
    \item \textbf{Accessibility:} The application should be accessible to users with disabilities, providing features like screen reader compatibility and keyboard navigation.
\end{itemize}

\subsection{Reliability Requirements}
\begin{itemize}
    \item \textbf{Uptime:} The application should have a high availability rate, with minimal downtime.
    \item \textbf{Error Handling:} The system should handle errors gracefully, providing clear messages to the user and options for recovery.
\end{itemize}

\subsection{Security Requirements}
\begin{itemize}
    \item \textbf{Data Privacy:} User data, including audio files and personal information, should be protected with strong encryption.
    \item \textbf{Authentication:} Secure user authentication methods should be in place to protect user accounts.
\end{itemize}

\subsection{Compatibility Requirements}
\begin{itemize}
    \item \textbf{Cross-Platform Support:} The application should be compatible with various operating systems, including Windows, macOS, and Linux.
    \item \textbf{File Format Support:} The application should support common audio file formats and produce PDFs that are compatible with standard PDF readers.
\end{itemize}

\subsection{Maintainability Requirements}
\begin{itemize}
    \item \textbf{Modularity:} The application code should be modular to facilitate easy updates and maintenance.
    \item \textbf{Documentation:} Comprehensive documentation should be provided for developers to understand the codebase and contribute to its development.
\end{itemize}

\subsection{Scalability Requirements}
\begin{itemize}
    \item \textbf{Load Handling:} The system should be able to handle multiple users and large audio files without significant performance degradation.
    \item \textbf{Future Expansion:} The architecture should allow for easy addition of new features and functionalities.
\end{itemize}

\section{Conclusion}
By addressing both the functional and non-functional requirements outlined above, this application can provide a robust, user-friendly solution for transcribing grand piano recordings into sheet music. This will not only meet the immediate needs of users but also ensure the application's longevity and adaptability to future demands.

\end{document}
